\documentclass[12pt, a4paper]{extarticle}
\usepackage{GOST}
\usepackage{array}
\usepackage{verbatim}
\usepackage[detect-all]{siunitx}
\usepackage{amsmath}
\usepackage{amssymb}
\usepackage[utf8]{inputenc}
\usepackage{hyperref}
\usepackage{tempora}

\makeatletter
\renewcommand\@biblabel[1]{#1.}
\makeatother

\usepackage{listings}
\lstset{ 
	language=Prolog,
	basicstyle=\small, 
	numbers=left, 
	numberstyle=\tiny,
	stepnumber=1,
	numbersep=5pt,
	showspaces=false,            
	showstringspaces=false,      
	showtabs=false,             
	frame=single,            % рисовать рамку вокруг кода
	tabsize=4,      
	commentstyle=\color{green},
	keywordstyle=\color{blue}\textbf,
	numberstyle=\scriptsize\color{gray}, % the style that is used for the line-numbers
	rulecolor=\color{black},
	captionpos=t,
	breaklines=true,         % автоматически переносить строки 
	breakatwhitespace=false, % переносить строки по пробелу
	%escapeinside={\#*}{*)} 
}




\begin{document}
	
\begin{table}[ht]
	\centering
	\begin{tabular}{|c|p{400pt}|} 
		\hline
		\begin{tabular}[c]{@{}c@{}} \includegraphics[scale=1]{source/b_logo.jpg} \\\end{tabular} &
		\footnotesize\begin{tabular}[c]{@{}c@{}}\textbf{Министерство~науки~и~высшего~образования~Российской~Федерации}\\\textbf{Федеральное~государственное~бюджетное~образовательное~учреждение}\\\textbf{~высшего~образования}\\\textbf{«Московский~государственный~технический~университет}\\\textbf{имени~Н.Э.~Баумана}\\\textbf{(национальный~исследовательский~университет)»}\\\textbf{(МГТУ~им.~Н.Э.~Баумана)}\\\end{tabular}  \\
		\hline
	\end{tabular}
\end{table}
\noindent\rule{\textwidth}{4pt}
\noindent\rule[14pt]{\textwidth}{1pt}
\hfill 
\noindent
\makebox{ФАКУЛЬТЕТ~}%
\makebox[\textwidth][l]{\underline{~«Информатика и системы управления»~~~~~~~~~~~~~~~~~~~~~~~~~~~~~~~~~}}%
\\
\noindent
\makebox{КАФЕДРА~}%
\makebox[\textwidth][l]{\underline{~«Программное обеспечение ЭВМ и информационные технологии»~}}%
\\

\begin{center}
	\vspace{1.5cm}
	{\bf\huge Отчёт\par}
	{\bf\Large по лабораторной работе № 17\par}
	\vspace{0.7cm}
\end{center}


\noindent
\makebox{\large{\bf Название:}~~~}
\makebox[\textwidth][l]{\large\underline{~Формирование эффективных программ на Prolog~}}\\

\noindent
\makebox{\large{\bf Дисциплина:}~~~}
\makebox[\textwidth][l]{\large\underline{~Функциональное и логическое программирование~}}\\

\vspace{1.5cm}
\noindent
\begin{tabular}{l c c c c c}
	Студент      & ~ИУ7-65Б~               & \hspace{2.5cm} & \hspace{2cm}                 & &  Д.В. Сусликов \\\cline{2-2}\cline{4-4} \cline{6-6} 
	\hspace{3cm} & {\footnotesize(Группа)} &                & {\footnotesize(Подпись, дата)} & & {\footnotesize(И.О. Фамилия)}
\end{tabular}

\noindent
\begin{tabular}{l c c c c}
	Преподаватель & \hspace{5cm}   & \hspace{2cm}                 & & ~~~~~~Н.Б. Толпинская~~~~~~\\\cline{3-3} \cline{5-5} 
	\hspace{3cm}  &                & {\footnotesize(Подпись, дата)} & & {\footnotesize(И.О. Фамилия)}
\end{tabular}

\vspace{0.6cm}
\begin{center}	
	\vfill
	\large \textit {Москва, 2021}
\end{center}

\thispagestyle {empty}
\pagebreak

\clearpage

\textbf{Задание}\par

\begin{enumerate}
	\item Максимум из двух чисел
	\begin{enumerate}
		\item без использования отсечения,
		\item с использованием отсечения;
	\end{enumerate}
	\item Максимум из трех чисел 
	\begin{enumerate}
		\item без использования отсечения,
		\item с использованием отсечения;
	\end{enumerate}
\end{enumerate}

Убедиться в правильности результатов.

Для каждого случая пункта 2 обосновать необходимость всех условий тела. 

Для одного из вариантов вопроса и каждого варианта задания 2 составить таблицу, отражающую конкретный порядок работы системы

\hfill

\textbf{Листинг:}

\begin{lstlisting}
domains
	num = integer.

predicates
	max(num, num, num).
	max_cut(num, num, num).

clauses
	max_cut(X1, X2, X1):- X1 >= X2, !.
	max_cut(_, X2, X2).	
	
	max(X1, X2, X1):- X1 >= X2.
	max(X1, X2, X1):- X2 > X1.
\end{lstlisting} \par
\newpage
\textbf{Листинг:}\par

\begin{lstlisting}
	domains
		num = integer.
	
	predicates	
		max3(num,num, num, num).
		max3_cut(num,num, num, num).
	
	clauses
		max3(X1, X2, X3, X1):- X1 >= X2, X1 >= X3.
		max3(X1, X2, X3, X2):- X2 > X1, X2 >= X3.
		max3(X1, X2, X3, X3):- X3 > X1, X3 > X2.
		
		max3_cut(X1, X2, X3, X1):- X1 > X2, X1 > X3, !.
		max3_cut(_, X2, X3, X2):- X2 > X3, !.
		max3_cut(_, _, X3, X3).
\end{lstlisting}
\newpage
\textbf{Таблица. }\par
2a) max3(2, 3, 1, Res)\par

\begin{table}[h!]
	\begin{tabular}{|l|l|l|l|}
		\hline
		\begin{tabular}[c]{@{}l@{}}№\\ шага\end{tabular} & \begin{tabular}[c]{@{}l@{}}Состояние резольвенты\\ и вывод\end{tabular}             & \begin{tabular}[c]{@{}l@{}}Для каких термов запускается\\ алгоритм унификации и каков\\ результат\end{tabular}                                                              & Дальнейшие действия                                                             \\ \hline
		0                                                & max3(2 , 3, 1, Res)                                                                 &                                                                                                                                                                             &                                                                                 \\ \hline
		1                                                & max3(2, 3, 1, Res)                                                                  & \begin{tabular}[c]{@{}l@{}}T1 = max3(2, 3, 1, Res)\\ T2 = max3(X1, X2, X3, X1)\\ Успех. Унифицируемы.\\ \\ Подстановка:\\ \{X1 = 2, X2 = 3, X3 = 1, X1 = Res\}\end{tabular} & \begin{tabular}[c]{@{}l@{}}Замена на тело\\ предложения\end{tabular}            \\ \hline
		2                                                & \begin{tabular}[c]{@{}l@{}}2  \textgreater{}= 3,\\ 2 \textgreater{}= 1\end{tabular} & \begin{tabular}[c]{@{}l@{}}2 \textgreater{}= 3\\ Ложь.\end{tabular}                                                                                                         & Откат                                                                           \\ \hline
		3                                                & max3(2, 3, 1, Res)                                                                  & \begin{tabular}[c]{@{}l@{}}T1 = max3(2, 3, 1, Res)\\ T2 = max3(X1, X2, X3, X2)\\ Успех. Унифицируемы.\\ \\ Подстановка:\\ \{X1 = 2, X2 = 3, X3 = 1, X2 = Res\}\end{tabular} & \begin{tabular}[c]{@{}l@{}}Замена на тело\\ предложения\end{tabular}            \\ \hline
		4                                                & \begin{tabular}[c]{@{}l@{}}3 \textgreater 2,\\ 3 \textgreater{}= 1\end{tabular}     & \begin{tabular}[c]{@{}l@{}}3 \textgreater 2\\ Верно.\end{tabular}                                                                                                           & \begin{tabular}[c]{@{}l@{}}Замена на тело\\ предложения\\ (пустое)\end{tabular} \\ \hline
		5                                                & 3 \textgreater{}= 1                                                                 & \begin{tabular}[c]{@{}l@{}}3 \textgreater 1\\ Верно\end{tabular}                                                                                                            & \begin{tabular}[c]{@{}l@{}}Замена на тело\\ предложения\\ (пустое)\end{tabular} \\ \hline
		6                                                & \begin{tabular}[c]{@{}l@{}}Резольвента пуста\\ Вывод Res = 3\end{tabular}           &                                                                                                                                                                             & Откат.                                                                          \\ \hline
                                                                                                              
	\end{tabular}
\end{table}

\newpage

\begin{table}[h!]
	\begin{tabular}{|l|l|l|l|}
		\hline
		7  & max3(2, 3, 1, Res)                                                           & \begin{tabular}[c]{@{}l@{}}T1 = max3(2, 3, 1, Res)\\ T2 = max3(X1, X2, X3, X3)\\ Успех. Унифицируемы.\\ \\ Подстановка:\\ \{X1 = 2, X2 = 3, X3 = 1, X3 = Res\}\end{tabular} & \begin{tabular}[c]{@{}l@{}}Замена на тело\\ предложения\end{tabular}         \\ \hline
		8  & \begin{tabular}[c]{@{}l@{}}1 \textgreater 2,\\ 1 \textgreater 3\end{tabular} & \begin{tabular}[c]{@{}l@{}}1 \textgreater 2\\ Неверно.\end{tabular}                                                                                                         & Откат.                                                                       \\ \hline
		9  & max3(2, 3, 1, Res)                                                           & \begin{tabular}[c]{@{}l@{}}T1 = max3(2, 3, 1, Res)\\ T2 = max3\_cut(...)\\ Неудача. Не унифицируемы.\end{tabular}                                                           & \begin{tabular}[c]{@{}l@{}}Переход к следующему\\ заголовку БЗ.\end{tabular} \\ \hline
		& ...                                                                          & ...                                                                                                                                                                         & ...                                                                          \\ \hline
		10 & max3(2, 3, 1, Res)                                                           & \begin{tabular}[c]{@{}l@{}}T1 = max3(2, 3, 1, Res)\\ T2 = max3\_cut(...)\\ Неудача. Не унифицируемы.\end{tabular}                                                           & \begin{tabular}[c]{@{}l@{}}Все предложения БЗ\\ пройдены.\end{tabular}       \\ \hline
		11 & Резольвента.                                                                 &                                                                                                                                                                             & \begin{tabular}[c]{@{}l@{}}Завершение работы\\ программы.\end{tabular}       \\ \hline
	\end{tabular}
\end{table}
\newpage

\textbf{Таблица. }\par
2b) max3\_cut(2, 3, 1, Res)\par

\begin{table}[h!]
	\begin{tabular}{|l|l|l|l|}
		\hline
		\begin{tabular}[c]{@{}l@{}}№\\ шага\end{tabular} & \begin{tabular}[c]{@{}l@{}}Состояние резольвенты\\ и вывод\end{tabular}            & \begin{tabular}[c]{@{}l@{}}Для каких термов запускается\\ алгоритм унификации и каков\\ результат\end{tabular}                                                                        & Дальнейшие действия                                                             \\ \hline
		0                                                & max3\_cut(2 , 3, 1, Res)                                                           &                                                                                                                                                                                       &                                                                                 \\ \hline
		1                                                & max3\_cut(2, 3, 1, Res)                                                            & \begin{tabular}[c]{@{}l@{}}T1 = max3\_cut(2, 3, 1, Res)\\ T2 = max3(...)\\ Неудача. Не унифицируемы.\end{tabular}                                                                     & \begin{tabular}[c]{@{}l@{}}Переход к следующему\\ заголовку БЗ.\end{tabular}    \\ \hline
		& ...                                                                                & ...                                                                                                                                                                                   & ...                                                                             \\ \hline
		2                                                & max3\_cut(2, 3, 1, Res)                                                            & \begin{tabular}[c]{@{}l@{}}T1 = max3\_cut(2, 3, 1, Res)\\ T2 = max3\_cut(X1, X2, X3, X1)\\ Успех. Унифицируемы.\\ \\ Подстановка:\\ \{X1 = 2, X2 = 3, X3 = 1, X1 = Res\}\end{tabular} & \begin{tabular}[c]{@{}l@{}}Замена на тело\\ предложения\end{tabular}            \\ \hline
		3                                                & \begin{tabular}[c]{@{}l@{}}2  \textgreater 3,\\ 2 \textgreater 1,\\ !\end{tabular} & \begin{tabular}[c]{@{}l@{}}2 \textgreater 3\\ Ложь.\end{tabular}                                                                                                                      & Откат                                                                           \\ \hline
		4                                                & max3\_cut(2, 3, 1, Res)                                                            & \begin{tabular}[c]{@{}l@{}}T1 = max3\_cut(2, 3, 1, Res)\\ T2 = max3\_cut(\_, X2, X3, X2)\\ Успех. Унифицируемы.\\ \\ Подстановка:\\ \{X1 = 2, X2 = 3, X3 = 1, X2 = Res\}\end{tabular} & \begin{tabular}[c]{@{}l@{}}Замена на тело\\ предложения\end{tabular}            \\ \hline
		5                                                & \begin{tabular}[c]{@{}l@{}}3 \textgreater 1,\\ !\end{tabular}                      & \begin{tabular}[c]{@{}l@{}}3 \textgreater 1\\ Верно.\end{tabular}                                                                                                                     & \begin{tabular}[c]{@{}l@{}}Замена на тело\\ предложения\\ (пустое)\end{tabular} \\ \hline
		6                                                & !                                                                                  & \begin{tabular}[c]{@{}l@{}}!\\ Истина.\end{tabular}                                                                                                                                   & \begin{tabular}[c]{@{}l@{}}Замена на тело\\ предложения\\ (пустое)\end{tabular} \\ \hline
		7                                                & \begin{tabular}[c]{@{}l@{}}Резольвента пуста\\ Вывод Res = 3\end{tabular}          &                                                                                                                                                                                       & Откат.                                                                          \\ \hline
	\end{tabular}
\end{table}
\newpage

\begin{table}[h!]
	\begin{tabular}{|l|l|l|l|}
		\hline
		8 & !                  & \begin{tabular}[c]{@{}l@{}}!\\ Завершение процедуры\end{tabular} & \begin{tabular}[c]{@{}l@{}}Замена на тело\\ предложения\\ (пустое)\end{tabular} \\ \hline
		9 & Резольвента пуста. &                                                                  & \begin{tabular}[c]{@{}l@{}}Завершения работы\\ программы\end{tabular}           \\ \hline
	\end{tabular}
\end{table}

\textbf{Ответы на вопросы}\par

\begin{enumerate} 
	
	\item[1)] В каком случае система запускает алгоритм унификации? (Как эту необходимость на формальном уровне распознает система?)
	
	Система запускает алгоритм унификации, когда резольвента не пуста.
	
	\item[2)] Каковы назначение и результат использования алгоритма унификации? 
	
	Алгоритм унификации необходим для того, чтобы подобрать знание, чтобы
	ответить на поставленный вопрос. Результатом работы алгоритма является
	значение переменной «неудача». Если неудача = 1, то унификация невозможна;
	если неудача = 0, то унификация прошла успешно, а побочным действием работы
	алгоритма является содержимое результирующей ячейки – результирующая
	подстановка.
	
	\item[3)] Какое первое состояние резольвенты?
	
	Вопрос. 
	
	\item[4)] Как меняется резольвента?
	
	Резольвента меняется в 2 этапа:
	\begin{itemize}
		\item Редукция (замена вопроса на тело правила, заголовок которого был успешно унифицирован);
		\item Применение подстановки.
	\end{itemize}
	
	\item[5)] В каких пределах программы уникальны переменные? 
	
	Именованные переменные уникальны в рамках предложения, анонимные --  везде.
		
	\item[6)] Как применяется подстановка, полученная с помощью алгоритма унификации?
	
	В результате подстановки связываются переменные, которые еще не были
	связаны. После связывания всех утверждений, будет напечатано значение
	связанных переменных.
	
	\item[7)] В каких случаях запускается механизм отката?
	
	В случае, когда унификация на текущем шаге завершается тупиковой
	ситуацией, или был получен ответ «да».	
\end{enumerate}

\end{document}