\documentclass[12pt, a4paper]{extarticle}
\usepackage{GOST}
\usepackage{array}
\usepackage{verbatim}
\usepackage[detect-all]{siunitx}
\usepackage{amsmath}
\usepackage{amssymb}
\usepackage[utf8]{inputenc}
\usepackage{hyperref}
\usepackage{tempora}

\makeatletter
\renewcommand\@biblabel[1]{#1.}
\makeatother

\usepackage{listings}
\lstset{ 
	language=Prolog,
	basicstyle = \ttfamily\color{blue},
	moredelim = [s][\color{black}]{(}{)},
	tabsize=4,
	literate =
	{:-}{{\textcolor{black}{:-}}}2
	{,}{{\textcolor{black}{,}}}1
	{.}{{\textcolor{black}{.}}}1
}



\begin{document}
	
\begin{table}[ht]
	\centering
	\begin{tabular}{|c|p{400pt}|} 
		\hline
		\begin{tabular}[c]{@{}c@{}} \includegraphics[scale=1]{source/b_logo.jpg} \\\end{tabular} &
		\footnotesize\begin{tabular}[c]{@{}c@{}}\textbf{Министерство~науки~и~высшего~образования~Российской~Федерации}\\\textbf{Федеральное~государственное~бюджетное~образовательное~учреждение}\\\textbf{~высшего~образования}\\\textbf{«Московский~государственный~технический~университет}\\\textbf{имени~Н.Э.~Баумана}\\\textbf{(национальный~исследовательский~университет)»}\\\textbf{(МГТУ~им.~Н.Э.~Баумана)}\\\end{tabular}  \\
		\hline
	\end{tabular}
\end{table}
\noindent\rule{\textwidth}{4pt}
\noindent\rule[14pt]{\textwidth}{1pt}
\hfill 
\noindent
\makebox{ФАКУЛЬТЕТ~}%
\makebox[\textwidth][l]{\underline{~«Информатика и системы управления»~~~~~~~~~~~~~~~~~~~~~~~~~~~~~~~~~}}%
\\
\noindent
\makebox{КАФЕДРА~}%
\makebox[\textwidth][l]{\underline{~«Программное обеспечение ЭВМ и информационные технологии»~}}%
\\

\begin{center}
	\vspace{1.5cm}
	{\bf\huge Отчёт\par}
	{\bf\Large по лабораторной работе № 19\par}
	\vspace{0.7cm}
\end{center}


\noindent
\makebox{\large{\bf Название:}~~~}
\makebox[\textwidth][l]{\large\underline{~Обработка списков на Prolog~}}\\

\noindent
\makebox{\large{\bf Дисциплина:}~~~}
\makebox[\textwidth][l]{\large\underline{~Функциональное и логическое программирование~}}\\

\vspace{1.5cm}
\noindent
\begin{tabular}{l c c c c c}
	Студент      & ~ИУ7-65Б~               & \hspace{2.5cm} & \hspace{2cm}                 & &  Д.В. Сусликов \\\cline{2-2}\cline{4-4} \cline{6-6} 
	\hspace{3cm} & {\footnotesize(Группа)} &                & {\footnotesize(Подпись, дата)} & & {\footnotesize(И.О. Фамилия)}
\end{tabular}

\noindent
\begin{tabular}{l c c c c}
	Преподаватель & \hspace{5cm}   & \hspace{2cm}                 & & ~~~~~~Н.Б. Толпинская~~~~~~\\\cline{3-3} \cline{5-5} 
	\hspace{3cm}  &                & {\footnotesize(Подпись, дата)} & & {\footnotesize(И.О. Фамилия)}
\end{tabular}

\vspace{0.6cm}
\begin{center}	
	\vfill
	\large \textit {Москва, 2021}
\end{center}

\thispagestyle {empty}
\pagebreak

\clearpage

\textbf{Задание}

Используя хвостовую рекурсию, разработать эффективную программу, позволяющую:
\begin{enumerate}
	\item Найти длину списка (по верхнему уровню);
	\item Найти сумму элементов числового списка 
	\item Найти сумму элементов числового списка, стоящих на нечетных позициях исходного списка (нумерация от 0) 
\end{enumerate}

Убедиться в правильности результатов. 

Для одного из вариантов вопроса и одного из заданий составить таблицу, отражающую конкретный порядок работы системы. 


\hfill

\textbf{Листинг:}

\begin{lstlisting}
	domains
	list = integer*
	
	predicates
		len(list, integer)
		len(list, integer, integer)
		
		sum(list, integer)
		sum(list, integer, integer)
		
		sum2(list, integer)
		sum2(list, integer, integer)
	
	clauses
		len([], R, R):- !.
		len([_|T], R, Res):- R1 = R + 1, len(T, R1, Res).
		len([H|T], Res):- len([H|T], 0, Res).
		
		sum([], R, R):- !.
		sum([H|T], R, Res):- R1 = R + H, sum(T, R1, Res).
		sum([H|T], Res):- sum([H|T], 0, Res).
		
		sum2([], R, R):- !.
		sum2([_, H|T], R, Res):- R1 = R + H, sum2(T, R1, Res).
		sum2([H|T], Res):- sum2([H|T], 0, Res).
		
	
	goal
		%len([1,2,3,4], Res).
		%sum([1,2,3,5], Res).
		sum2([1,2,3,11], Res).
\end{lstlisting}
\par
Результаты работы:\par
\begin{figure}[h!]
	\begin{minipage}[h]{0.31\linewidth}
		\center{\includegraphics[width=0.33\linewidth]{source/1.png} \\ Пример len}	
	\end{minipage}
	\hfill
	\begin{minipage}[h]{0.31\linewidth}
		\center{\includegraphics[width=0.33\linewidth]{source/2.png} \\ Пример sum}	
	\end{minipage}
	\hfill
	\begin{minipage}[h]{0.31\linewidth}
		\center{\includegraphics[width=0.33\linewidth]{source/3.png} \\ Пример sum2}	
	\end{minipage}
\end{figure}\par

\newpage
\textbf{Приведем таблицу для нахождения суммы. }\par
sum([1,2,4], Res)\par
Текст процедуры:
\begin{lstlisting}
1:		sum([], R, R):- !.
2:		sum([H|T], R, Res):- R1 = R + H, sum(T, R1, Res).
3:		sum([H|T], Res):- sum([H|T], 0, Res).
\end{lstlisting}\par

\begin{table}[h!]
	\begin{tabular}{|l|l|l|l|}
		\hline
		№ шага & \begin{tabular}[c]{@{}l@{}}Текущая \\ резольвента - ТР\end{tabular}          & \begin{tabular}[c]{@{}l@{}}ТЦ, выбираемые правила:\\ сравниваемые термы,\\ подстановка\end{tabular}                                        & \begin{tabular}[c]{@{}l@{}}Дальнейшие действия\\ с комментариями\end{tabular}    \\ \hline
		1      & sum({[}1,2{]}, 0, Res).                                                      & ТЦ:  sum({[}1,2{]}, 0, Res)                                                                                                                & \begin{tabular}[c]{@{}l@{}}Поиск знания\\ с начала БЗ\end{tabular}               \\ \hline
		& sum({[}1,2{]}, 0, Res).                                                      & \begin{tabular}[c]{@{}l@{}}ПР1:\\ {[}{]} = {[}1,2{]}\\ R = 0\\ R = Res\\ Неудача\end{tabular}                                              & \begin{tabular}[c]{@{}l@{}}Переход к\\ следующему\\ заголовку БЗ\end{tabular}    \\ \hline
		& sum({[}1,2{]}, 0, Res).                                                      & \begin{tabular}[c]{@{}l@{}}ПР2:\\ H|T{]} = {[}1,2{]}\\ R = 0\\ Res = Res\\ \\ Успех\\ H = 1\\ T = {[}2{]}\\ R = 0\\ Res = Res\end{tabular} & \begin{tabular}[c]{@{}l@{}}Тело ПР2\\ заменяет цель\\ в резольвенте\end{tabular} \\ \hline
		2      & \begin{tabular}[c]{@{}l@{}}R1 = 0 + 1, \\ sum({[}2{]}, R1, Res)\end{tabular} & R1 = 1                                                                                                                                     & \begin{tabular}[c]{@{}l@{}}R1 = 0 + 1 заменяется\\ на пустое тело\end{tabular}   \\ 
	\end{tabular}
\end{table}

\newpage

	\begin{table}[h!]
		\begin{tabular}{|l|l|l|l|}
			\hline
			3 & sum({[}2{]}, 1, Res)                                                        & ТЦ: sum({[}2{]},1,Res)                                                                                                                  & \begin{tabular}[c]{@{}l@{}}Поиск знания с\\ начала БЗ\end{tabular}                             \\ \hline
			& sum({[}2{]}, 1, Res)                                                        & \begin{tabular}[c]{@{}l@{}}ПР1:\\ {[}{]} = {[}2{]}\\ R = 1\\ R = Res\\ Неудача\end{tabular}                                             & \begin{tabular}[c]{@{}l@{}}Переход к\\ следующему\\ заголовку БЗ\end{tabular}                  \\ \hline
			& sum({[}2{]}, 1, Res).                                                       & \begin{tabular}[c]{@{}l@{}}ПР2:\\ H|T{]} = {[}2{]}\\ R = 1\\ Res = Res\\ \\ Успех\\ H = 2\\ T = {[}{]}\\ R = 1\\ Res = Res\end{tabular} & \begin{tabular}[c]{@{}l@{}}Тело ПР2\\ заменяет цель\\ в резольвенте\end{tabular}               \\ \hline
			4 & \begin{tabular}[c]{@{}l@{}}R1 = 1 + 2, \\ sum({[}{]}, R1, Res)\end{tabular} & R1 = 3                                                                                                                                  & \begin{tabular}[c]{@{}l@{}}R1 = 1 + 2 заменяется\\ на пустое тело в\\ резольвенте\end{tabular} \\ \hline
			5 & sum({[}{]}, 3, Res)                                                         & ТЦ: sum({[}{]},3,Res)                                                                                                                   & \begin{tabular}[c]{@{}l@{}}Поиск знания с\\ начала БЗ\end{tabular}                             \\ \hline
			& sum({[}{]}, 3, Res)                                                         & \begin{tabular}[c]{@{}l@{}}ПР1:\\ {[}{]} = {[}{]}\\ R = 3\\ R = Res\\ \\ Успех\\ {[}{]} = {[}{]}\\ R = 3\\ R = Res = 3\end{tabular}     & \begin{tabular}[c]{@{}l@{}}Тело ПР1\\ заменяет цель\\ в резольвенте\end{tabular}               \\ 
		\end{tabular}
	\end{table}	

	\newpage
	
	\begin{table}[h!]
		\begin{tabular}{|l|l|l|l|}
			\hline
			6 & !                                                           & \begin{tabular}[c]{@{}l@{}}!.\\ Истина\end{tabular}              & \begin{tabular}[c]{@{}l@{}}Отсечение.\\ ! заменяется на \\ пустое в резольвенте\end{tabular} \\ \hline
			7 & Резольвента пуста                                           &                                                                  & \begin{tabular}[c]{@{}l@{}}Выводится Res = 3\\ Откат\end{tabular}                            \\ \hline
			8 & !                                                           & \begin{tabular}[c]{@{}l@{}}!\\ Завершение \\ процедуры\end{tabular} & \begin{tabular}[c]{@{}l@{}}! заменяется на \\ пустое в резольвенте\end{tabular}              \\ \hline
			7 & \begin{tabular}[c]{@{}l@{}}Резольвента\\ пуста\end{tabular} &                                                                  & \begin{tabular}[c]{@{}l@{}}Завершение работы\\ программы\end{tabular}                        \\ \hline
		\end{tabular}
	\end{table}

\textbf{Вывод}\par
Эффективность работы системы может быть достигнута за счет хвостовой
рекурсии и использования отсечения в тех
случаях, где заведомо известна единственность ответа на вопрос.

\newpage

\textbf{Ответы на вопросы}
\begin{itemize}
	\item[1)] Что такое рекурсия? Как организуется хвостовая рекурсия в Prolog? Как организовать выход из рекурсии в Prolog?\par	
	Рекурсия – определение объекта через ссылку на самого себя. Один из
	способов организации повторных вычислений. Для организации хвостовой рекурсии необходимо, чтобы рекурсивный вызов был последним в теле рекурсивного правила, и не оставалось других точек выбора. Выход из рекурсии осуществляется либо достижением базиса рекурсии, либо условием в теле правила.
	
	\item[2)] Какое первое состояние резольвенты?\par
	Исходная резольвента содержит вопрос.
	
	\item[3)] В каких пределах программы уникальны переменные?\par
	Именованные переменные уникальны в рамках предложения, анонимные - уникальны везде.
	
	\item[4)] В какой момент, и каким способом системе удается получить доступ к голове списка?\par	
	Получить доступ к голове списка можно при его унификации со списком
	вида [H | T], где H - голова, T - хвост.
	
	\item[5)] Каково назначение использования алгоритма унификации?\par
	Алгоритм унификации необходим для того, чтобы подобрать знание, положительно отвечающее на поставленный вопрос.
	
	\item[6)] Каков результат работы алгоритма унификации?\par
	Результатом работы алгоритма является значение переменной «неудача».
	Если неудача = 1, то унификация невозможна; если неудача = 0, то унификация
	прошла успешно, а побочным действием работы алгоритма является содержимое
	результирующей ячейки – результирующая подстановка.
	
	\item[7)] Как формируется новое состояние резольвенты?\par
	Резольвента меняется в 2 этапа:
	\begin{itemize}
		\item Редукция (замена вопроса на тело правила, заголовок которого был успешно унифицирован);
		\item Применение подстановки.
	\end{itemize}

	\item[8)] Как применяется подстановка, полученная с помощью алгоритма
	унификации – как глубоко?\par
	В результате подстановки связываются переменные, которые еще не были
	связаны. После связывания всех утверждений, будет напечатано значение
	связанных переменных.
	
	\item[9)] В каких случаях запускается механизм отката?\par
	В случае, когда унификация на текущем шаге завершается тупиковой
	ситуацией, или был получен ответ «да».
	
	\item[10)]  Когда останавливается работа системы? Как это определяется на формальном уровне?\par
	Когда резольвента пуста и все указатели находятся в конце БЗ.
\end{itemize}

\end{document}